\documentclass[12pt, a4paper, titlepage]{report}

    %Germanish stuff
    \usepackage[utf8]{inputenc}
    \usepackage[T1]{fontenc}
    \usepackage{ngerman}

    % Seitenränder
    \usepackage{geometry}
    \geometry{a4paper, top=25mm, left=25mm, right=45mm, bottom=20mm, headsep=10mm, footskip=10mm}

    % 1 1/2 Zeilig
    \usepackage[onehalfspacing]{setspace}

    % Change Font to Arial
    \renewcommand{\familydefault}{\sfdefault}
    \usepackage{uarial}

    % Kopfzeilen, Nummerierung aus für Seiten 1+2
    \usepackage{fancyhdr}
    \lhead{Julian Tölle}
    \chead{\thepage}
    \rhead{Facharbeit}
    \cfoot{}
    \renewcommand{\headrulewidth}{0.4pt}

    \pagestyle{fancy}
    % Fix for Chapter pages:
    \usepackage{etoolbox}
    \patchcmd{\chapter}{\thispagestyle{plain}}{\thispagestyle{fancy}}{}{}

    \usepackage{bibgerm}
    \usepackage[backend=biber,
    			style=authoryear-icomp,
    			sortlocale=de_DE]{biblatex}
    \usepackage[babel, german=guillemets]{csquotes}
	\addbibresource{bibindex.bib}
	

	%\newcommand{\inlinecode}[1]{\texttt{#1}}

\begin{document}

    \begin{titlepage}
        \title{Verspricht die Sell-In-May-Anlagestrategie Erfolg?\\
                –\\
                Entwicklung eines Algorithmus zur Lösung dieses Optimierungsproblemes
                und der Datenanalyse anhand des Beispiels der Dax-Börsendaten}
        \author{Julian Tölle\\
                Ev. Gymnasium Werther\\
                Informatik-Grundkurs\\
                Herr Möllenbrock}
        \date{Schuljahr 2014/15}

        \maketitle
    \end{titlepage}

    \tableofcontents
    % No header for ToC
    \thispagestyle{empty}

    % Correct page numbering
    \setcounter{page}{2}

    \chapter{Einführung ins Thema}
        
        \section{Der Deutsche Aktienindex}
            Der DAX, kurz für Deutscher Aktienindex, wurde im Jahr 1988 erstmalig veröffentlicht,
            und
            % Was ist der Dax
            % Abgrenzung zu anderen Aktienindizes
            % Gründung
            % Handelsort
            % Auswahl der Wertpapiere

        \section{Sell-in-May als Anlagestrategie}
            \begin{quote}
                Sell in May and go away, but remember to come back in September.
            \end{quote}
            So lautet eine alte Börsenweisheit die sich darauf bezieht dass es
            in den Wintermonaten eine erheblich höhere Rendite für Anlagen gibt
            als in den Sommermonate. Nach ihr ist es am effektivsten im Mai alle
            aktuell gehaltenen Aktien zu verkaufen und im September wieder
            einzukaufen, um so den durchschnittlichen Wertverlust der Aktien im
            Sommer abzuwarten und dann wenn die Kurse wieder besser werden
            günstig einzukaufen.

            Der Effekt wurde nur geringfügig Wissenschaftlich untersucht, eine
            Studie ist allerdings zu dem Urteil gekommen dass er in 36 von 37
            untersuchten Ländern nachweisbar war\footcite{bouman2002halloween}. % Näher darauf eingehen
            % Sell in May and go away but remember to come back in september
            % Bisherige Studien zu dem Thema

        \section{Problemstellung}
            Die Zeitspanne eines Monats als Zeitraum für den Verkauf und Einkauf
            der Aktien ist sehr grob. Er gibt keinen Aufschluss über die besten
            Tageskombinationen sondern nur eine grobe Richtung bei der es zu
            unsicher ist ihr blind zu folgen. Und wo genau der maximale Profit
            steckt weiß auch so keiner richtig. Es wird also Zeit sich einmal
            anzugucken an welchen Tagen im Mai und September man historisch
            gesehen am besten hätte handeln sollen um den maximalen Profit
            herauszuholen. Beispielsweise verwende ich hier den DAX da er der
            bekannteste und meistgenutzte Indikator für die deutsche
            Aktienlandschaft ist.
            % Anwendung von Sell-in-May auf den Dax
            % Optimale Tage für die letzten Jahre
            % -> Optimierungsproblem

    \chapter{Anforderungsanalyse}

    \chapter{Entwicklung des Algorithmus}

        \section{Algorithmische Grundidee}
            Der Algorithmus besteht auf der Grundidee des Bruteforcing.
            Bruteforcing besteht auf der Idee dass man, obwohl es keinen
            effizienten, und möglicherweise komplizierteren, Algorithmus gibt, man
            einfach alle Optionen "ausprobiert" und die beste auswählt. Dies kann
            allerdings auch dazu führen dass die Laufzeit des Algorithmus bei großen
            Datenmengen weit über der eines effizienteren Algorithmus liegen kann.
            Da sich die Fragestellung allerdings nur im Bereich von wenigen Millionen
            Rechnungen\footnote{30 (Tage im Mai) * 31 (Tage im September) * 5500
            (Tageskurse) = 5.115.000} befindet, ist Bruteforcing eine brauchbare
            Alternative zu der Entwicklung eines alternativen Algorithmus.
   			Für eine extensive Nutzung des entwickelten Programms allerdings sollte
   			er noch weiter optimiert werden.

            Zu Beginn wird eine Liste, bestehend aus jeder durchzurechnenden Option,
            erstellt, jedes Element dieser besteht aus zwei Tagen. Der erste Tag 
            kommt aus der Zeitspanne zum Verkauf, hier also aus dem Mai, der zweite
            Tag kommt aus der Zeitspanne zum Einkauf, hier also aus dem September.
            Für jede mögliche Kombination der Elemente dieser beiden Listen gibt es
            einen Eintrag in der neu erstellten Liste.

            Nun wird für jede Kombination ihrer Rentabilität in Form eines
            Performance Indexes berechnet. Dies geschieht indem mit Hilfe eines
            simulierten Depots der Weg des Geldes nachgestellt wird.
            Am Anfang gibt es eine bestimmte Geldmenge, und nun werden alle
            geladenen Tageskurse des DAX chronologisch abgelaufen.
            Falls sich das Datum des Tageskurses zwischen Verkauf
            und Kauf befindet, werden alle noch im Depot vorhandenen Aktien
            verkauft, und falls es sich außerhalb dieser Zeitspannen befindet
            wird alles vorhandene Geld genutzt um möglichst viele Aktien zu
            kaufen. Dieses lazy-selling, das heißt zum nächstmöglichen Termin nach
            Beginn der Zeiträume, erspart eine komplizierte Fehlerbehebung falls
            es zu den genauen Tagen der Kombination keine Kurse gibt.

            Nach durchlaufen aller Tageskurse werden alle Aktien zum Kurs des
            letzten Tages verkauft und das nun vorhandene Geld mit dem
            Startbetrag verglichen:
            \begin{equation}
                performanceIndex = \frac{Geld_{Ende}}{Geld_{Anfang}}
            \end{equation}
            Nachdem dieser für alle möglichen Kombination berechnet wurde, wird
            das Optimum ermittelt und die korrespondierende Tageskombination
            abgespeichert.

        \section{Erläuterung der Implementation}
            %

    \chapter{Evaluation der Ergebnisse}

        \section{Präsentation der Ergebnisse}
            % 14-Fach
            % Vergleich mit pI von anderen Tageskombinationen

        \section{Bedeutung für Anlagestrategien}
            % Starke abhängigkeit von Börsencrashs
            % Wie stark verändern sich die optimalen Tage jedes Jahr?

    \chapter{Fazit}

    \chapter{Schlusserklärung}
        Ich erkläre, dass ich die Facharbeit ohne fremde Hilfe angefertigt habe
        und nur die im Literaturverzeichnis angeführten Quellen und Hilfsmittel benutzt habe.

    

\end{document}
