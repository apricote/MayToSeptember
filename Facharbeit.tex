\documentclass[12pt, a4paper, titlepage]{report}

    %Germanish stuff
    \usepackage[utf8x]{inputenc}
    \usepackage[T1]{fontenc}
    \usepackage{ngerman}

    % Seitenränder
    \usepackage{geometry}
    \geometry{a4paper, top=25mm, left=25mm, right=45mm, bottom=20mm, headsep=10mm, footskip=10mm}

    % 1 1/2 Zeilig
    \usepackage[onehalfspacing]{setspace}

    % Change Font to Arial
    \renewcommand{\familydefault}{\sfdefault}
    \usepackage{uarial}

    % Kopfzeilen, Nummerierung aus für Seiten 1+2
    \usepackage{fancyhdr}
    \lhead{Julian Tölle}
    \chead{\thepage}
    \rhead{Facharbeit}
    \cfoot{}
    \renewcommand{\headrulewidth}{0.4pt}

    \pagestyle{fancy}
    % Fix for Chapter pages:
    \usepackage{etoolbox}
    \patchcmd{\chapter}{\thispagestyle{plain}}{\thispagestyle{fancy}}{}{}

\begin{document}

    \begin{titlepage}
        \title{Entwicklung eines Algorithmus zur Lösung eines Optimierungsproblemes und der Datenanalyse anhand des Beispiels der Dax-Börsendaten}
        \author{Julian Tölle\\
                Ev. Gymnasium Werther\\
                Informatik-Grundkurs\\
                Herr Möllenbrock}
        \date{Schuljahr 2014/15}

        \maketitle
    \end{titlepage}

    \tableofcontents
    % No header for ToC
    \thispagestyle{empty}

    % Correct page numbering
    \setcounter{page}{2}

    \chapter{Einführung ins Thema}
        
        \section{Der Dax}
            % Was ist der Dax
            % Abgrenzung zu anderen Aktienindizes
            % Gründung
            % Handelsort
            % Auswahl der Wertpapiere

        \section{Sell-in-May}
            % Sell in May and go away but remember to come back in september
            % Bisherige Studien zu dem Thema

        \section{Problemstellung}
            % Anwendung von Sell-in-May auf den Dax
            % Optimale Tage für die letzten Jahre
            % -> Optimierungsproblem

    \chapter{Anforderungsanalyse}

    \chapter{Entwicklung des Algorithmus}

        \section{Grundidee}
            % Bruteforcing
            % performanceIndex

        \section{Implementation}
            %

    \chapter{Evaluation der Ergebnisse}

        \section{Präsentation der Ergebnisse}
            % 14-Fach
            % Vergleich mit pI von anderen Tageskombinationen

        \section{Bedeutung für Anlagestrategien}
            % Starke abhängigkeit von Börsencrashs
            % Wie stark verändern sich die optimalen Tage jedes Jahr?

    \chapter{Fazit}

    \chapter{Schlusserklärung}
        Ich erkläre, dass ich die Facharbeit ohne fremde Hilfe angefertigt habe
        und nur die im Literaturverzeichnis angeführten Quellen und Hilfsmittel benutzt habe.

\end{document}
