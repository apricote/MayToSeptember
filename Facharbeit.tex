\documentclass[12pt, a4paper, titlepage]{report}

    %Germanish stuff
    \usepackage[utf8]{inputenc}
    \usepackage[T1]{fontenc}
    \usepackage{ngerman}

    % Seitenränder
    \usepackage{geometry}
    \geometry{a4paper, top=25mm, left=25mm, right=45mm, bottom=20mm, headsep=10mm, footskip=10mm}

    % 1 1/2 Zeilig
    \usepackage[onehalfspacing]{setspace}

    % Change Font to Arial
    \renewcommand{\familydefault}{\sfdefault}
    \usepackage{uarial}

    % Kopfzeilen, Nummerierung aus für Seiten 1+2
    \usepackage{fancyhdr}
    \lhead{Julian Tölle}
    \chead{\thepage}
    \rhead{Facharbeit}
    \cfoot{}
    \renewcommand{\headrulewidth}{0.4pt}

    \pagestyle{fancy}
    % Fix for Chapter pages:
    \usepackage{etoolbox}
    \patchcmd{\chapter}{\thispagestyle{plain}}{\thispagestyle{fancy}}{}{}

    %\usepackage{bibgerm}
    \usepackage[backend=biber,
    			style=verbose,
    			sortlocale=de_DE,
    			language=german]{biblatex}
    \usepackage[babel, german=guillemets]{csquotes}
	\addbibresource{bibindex.bib}
	

	%\newcommand{\inlinecode}[1]{\texttt{#1}}

\begin{document}

    \begin{titlepage}
        \title{Verspricht die Sell-In-May-Anlagestrategie Erfolg?\\
                –\\
                Entwicklung eines Algorithmus zur Lösung dieses Optimierungsproblemes
                und der Datenanalyse anhand des Beispiels der Dax-Börsendaten}
        \author{Julian Tölle\\
                Ev. Gymnasium Werther\\
                Informatik-Grundkurs\\
                Herr Möllenbrock}
        \date{Schuljahr 2014/15}

        \maketitle
    \end{titlepage}

    \tableofcontents
    % No header for ToC
    \thispagestyle{empty}

    % Correct page numbering
    \setcounter{page}{2}

    \chapter{Einführung ins Thema}
        
        \section{Der Deutsche Aktienindex}
            Der DAX, kurz für Deutscher Aktienindex, wurde im Jahr 1988 erstmalig
            veröffentlicht, und beinhaltet die 30 größten und umsatzstärksten Aktien
            Deutschlands. Er basiert auf den Kursen der Frankfurter Wertpapierbörse,
            der größten deutschen Börse.
            
            Damit ein Unternehmen in den DAX kommen kann muss es mehrere Kriterien
            erfüllen. Zuerst muss das Unternehmen im Prime Standard der Frankfurter Börse
            gelistet sein, dies ist ein stärker regulierter Bereich der Börse der
            hohe Transparenzvorraussetzungen erfüllen muss
            \footcite{deutscheboersePrimeStandard}.
            Das nächste Kriterium ist dass das Unternehmen durchgehend bei Xetra gehandelt
            wird. Xetra ist die elektrische Handelssystem das an der Frankfurter Börse
            verwendet wird. Es übernimmt die gesamte Verwaltung des Handels sodass die
            Aufträge schneller als manuell abgearbeitet werden können und in hoher
            Frequenz neue Kurse bereitgestellt werden können\footcite{boerseFrankfurtXetra}.
            Das letzte Bedingung ist ein Streubesitz von über 10\%. Als Streubesitz werden
            wird der Teil der Aktien einer AG genannt, die nicht als große Pakete wenigen
            Investoren und Besitzern gehören, sondern frei in der Börse gehandelt werden
            \footcite{gablerFreefloat}.  
            
            Der DAX ist, anders als die meisten anderen großen
            Aktienindizes (Dow Jones Industrial Average, FTSE, Nikkei 225) ein
            Performance Index, was heißt dass alle Dividendenauszahlungen von Aktien
            diesem Index in den Index reinvestiert werden. Normalerweise, wenn eine Aktie
            Dividenden ausschüttet sinkt der Wert der Aktie um den Betrag der Dividende,
            ein Kursindex würde also nach der Dividendenausschüttung an Punkten verlieren,
            der DAX allerdings nicht, da der Dividendenabschlag durch die positive
            Bewertung der Auszahlung ausgeglichen wird.
            
            % Was ist der Dax
            % Abgrenzung zu anderen Aktienindizes
            % Gründung
            % Handelsort
            % Auswahl der Wertpapiere

        \section{Sell-in-May als Anlagestrategie}
            \begin{quote}
                Sell in May and go away, but remember to come back in September.
            \end{quote}
            So lautet eine alte Börsenweisheit die sich darauf bezieht dass es
            in den Wintermonaten eine erheblich höhere Rendite für Anlagen gibt
            als in den Sommermonate. Nach ihr ist es am effektivsten im Mai alle
            aktuell gehaltenen Aktien zu verkaufen und im September wieder
            einzukaufen, um so den durchschnittlichen Wertverlust der Aktien im
            Sommer abzuwarten und dann wenn die Kurse wieder besser werden
            günstig einzukaufen.

            Der Effekt wurde nur geringfügig wissenschaftlich untersucht, eine
            Studie ist allerdings zu dem Urteil gekommen dass er in 36 von 37
            untersuchten Ländern nachweisbar war\footcite{bouman2002halloween}.
            Um dies zu belegen, wurde die Rentabilität einer ''Buy and Hold Strategy``
            mit der Sell-in-May Anlagestrategie für alle untersuchten Märkte über
            mehrere Jahre verglichen. Dies ergab, dass das Sell-in-May Verfahren 
            durchschnittlich 1,5\%, im Hinblick auf die jährliche Rendite, besser
            war als die Aktien dass ganze Jahr zu halten\footnote{Vgl. Bouman und Jacobsen 2002, Seite 14 Tabelle A1}.
            % Sell in May and go away but remember to come back in september
            % Bisherige Studien zu dem Thema

        \section{Problemstellung}
            Die Zeitspanne eines Monats als Zeitraum für den Verkauf und Einkauf
            der Aktien ist sehr grob. Er gibt keinen Aufschluss über die besten
            Tageskombinationen sondern nur eine grobe Richtung bei der es zu
            unsicher ist ihr blind zu folgen. Und wo genau der maximale Profit
            steckt weiß auch so keiner richtig. Magere 1,5\% verspricht die
            vorher erwähnte Studie. Es wird Zeit, dass man diese Rendite maximiert.
            Da Marktvorhersagen allerdings nur schwer möglich sind, richten wir
            in dieser Facharbeit unser Augenmerk auf die vergangenen Jahre und 
            schreiben einen Algorithmus der die, historisch gesehen, optimalsten Tage
            für den Kauf und Verkauf der Aktien berechnet. In der Auswertung dieser
            Daten wird dann eine mögliche Tendenz gesucht, die die zukünftigen
            Investitionen vereinfachen soll.
            Als zu betrachtenden Kurs verwende ich den DAX, da er der bekannteste und
            meistgenutzte Indikator für die deutsche Aktienlandschaft ist.
            % Anwendung von Sell-in-May auf den Dax
            % Optimale Tage für die letzten Jahre
            % -> Optimierungsproblem

    \chapter{Anforderungsanalyse}

    \chapter{Entwicklung des Algorithmus}

        \section{Algorithmische Grundidee}
            Der Algorithmus besteht auf der Grundidee des Bruteforcing.
            Bruteforcing besteht auf der Idee dass man, obwohl es keinen
            effizienten, und möglicherweise komplizierteren, Algorithmus gibt, man
            einfach alle Optionen "ausprobiert" und die beste auswählt. Dies kann
            allerdings auch dazu führen dass die Laufzeit des Algorithmus bei großen
            Datenmengen weit über der eines effizienteren Algorithmus liegen kann.
            Da sich die Fragestellung allerdings nur im Bereich von wenigen Millionen
            Rechnungen\footnote{30 (Tage im Mai) * 31 (Tage im September) * 4370
            (Tageskurse) = 4.064.100} befindet, ist Bruteforcing eine brauchbare
            Alternative zu der Entwicklung eines alternativen Algorithmus.
   			Für eine extensive Nutzung des entwickelten Programms allerdings sollte
   			er noch weiter optimiert werden.

            Zu Beginn wird eine Liste, bestehend aus jeder durchzurechnenden Option,
            erstellt, jedes Element dieser besteht aus zwei Tagen. Der erste Tag 
            kommt aus der Zeitspanne zum Verkauf, hier also aus dem Mai, der zweite
            Tag kommt aus der Zeitspanne zum Einkauf, hier also aus dem September.
            Für jede mögliche Kombination der Elemente dieser beiden Listen gibt es
            einen Eintrag in der neu erstellten Liste.

            Nun wird für jede Kombination ihrer Rentabilität in Form eines
            Performance Indexes berechnet. Dies geschieht indem mit Hilfe eines
            simulierten Depots der Weg des Geldes nachgestellt wird.
            Am Anfang gibt es eine bestimmte Geldmenge, und nun werden alle
            geladenen Tageskurse des DAX chronologisch abgelaufen.
            Falls sich das Datum des Tageskurses zwischen Verkauf
            und Kauf befindet, werden alle noch im Depot vorhandenen Aktien
            verkauft, und falls es sich außerhalb dieser Zeitspannen befindet
            wird alles vorhandene Geld genutzt um möglichst viele Aktien zu
            kaufen. Dieses lazy-selling, das heißt zum nächstmöglichen Termin nach
            Beginn der Zeiträume, erspart eine komplizierte Fehlerbehebung falls
            es zu den genauen Tagen der Kombination keine Kurse gibt.

            Nach durchlaufen aller Tageskurse werden alle Aktien zum Kurs des
            letzten Tages verkauft und das nun vorhandene Geld mit dem
            Startbetrag verglichen:
            \begin{equation}
                performanceIndex = \frac{Geld_{Ende}}{Geld_{Anfang}}
            \end{equation}
            Nachdem dieser für alle möglichen Kombination berechnet wurde, wird
            das Optimum ermittelt und die korrespondierende Tageskombination
            abgespeichert.

        \section{Erläuterung der Implementation}
            Nachdem nun der Ablauf des Algorithmus grob umrissen wurde, wird nun
            auf die genaue Implementation eingegangen. Alle folgenden Codelistings
            sind der Klasse com.toelle.maytoseptember.controller.StockOptimizer entnommen.
            
            

    \chapter{Evaluation der Ergebnisse}

        \section{Präsentation der Ergebnisse}
            % 14-Fach
            % Vergleich mit pI von anderen Tageskombinationen

        \section{Bedeutung für Anlagestrategien}
            % Starke abhängigkeit von Börsencrashs
            % Wie stark verändern sich die optimalen Tage jedes Jahr?

    \chapter{Fazit}

    \chapter{Schlusserklärung}
        Ich erkläre, dass ich die Facharbeit ohne fremde Hilfe angefertigt habe
        und nur die im Literaturverzeichnis angeführten Quellen und Hilfsmittel benutzt habe.
        
    \printbibliography

\end{document}
